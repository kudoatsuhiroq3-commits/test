\documentclass[a4paper,11pt]{jsarticle}


% 数式
\usepackage{amsmath,amsfonts}
\usepackage{amssymb}
\usepackage{txfonts}
\usepackage{bm}
\usepackage{physics}
\usepackage{mathtools}
\numberwithin{equation}{section}
\usepackage{empheq}
\newcommand{\bvec}[1]{\mbox{\boldmath $#1$}}

% 画像
\usepackage{graphicx}
\usepackage[dvipdfmx]{color}
\usepackage{multicol}
\usepackage{wrapfig}

%化学系
\usepackage{chemfig}

%その他
\usepackage{siunitx}
\usepackage{float}
\usepackage{tikz}
\usepackage{circuitikz}
\usepackage{diagbox}


\begin{document}

\title{スピン軌道相互作用の導出}
\author{工藤英鉄(理学部物理学科4年)}
\date{}
\maketitle

\setcounter{section}{0}

\begin{abstract}
  このノートではディラック方程式を用いてスピン軌道相互作用を導出します.\\
  基本的には\cite{II}を参考にして書いたが,適宜他の文献も参考にした.
\end{abstract} 

\section{はじめに}
スピン軌道相互作用(Spin orbit interaction)は電子の持つスピン$\bvec{s}$と軌道角運動量$\bvec{l}$間における
相互作用であり,一般的に
\begin{align}
  \mathcal{H} _{so} =\lambda \bvec{s}\cdot \bvec{l}
\end{align}
の形で書かれる.ここでは$\lambda$は相互作用の大きさを特徴付けるある定数である.
この相互作用は古典的には次のように理解できる(以下\cite{soi}を参照,またこの節では$\hbar=1$として考える).

\begin{wrapfigure}{r}[0pt]{0.4\textwidth}
  \centering
    \includegraphics[width=6cm]{C0_image1.png}
  \caption{原子核中心の描像(左図)と電子中心の描像(右図)}
  \label{fig:1}
 \end{wrapfigure}

原子核の周りを円運動している電子を考える.
これは電子の立場で考えると原子が電子の周りを円運動していることになる(図1参照).
この描像で考えると電子は原子核が成す円電流の中心にいることになるので,
電子は円電流が作る磁場を感じて電磁相互作用を起こす.
この磁場は原子核の電荷を$Ze$, 電子は原子核から半径$r$の周りを速度$\bvec{v}$で周っているとすると,
ビオ・サヴァールの法則を用いて次のようになる.
\begin{align}
  \bvec{B}_{\text{eff}} =\frac{\mu_0}{4\pi}Ze\frac{\bvec{r}×\bvec{v}}{r^3} 
  = \frac{\mu_0}{4\pi}\frac{Ze}{m}\frac{ \bvec{l}}{r^3}
\end{align}
ここでは電子の軌道角運動量$\bvec{l}(=m\bvec{r}×\bvec{v})$を用いて変形した.

ここで電子もスピン運動により円運動をして磁気双極子を形成しているように見れる
\footnote{スピンは電子が円運動をしているという説明は正しくないが直観的に分かりやすいのでよくこのように例えられる.
この描像のせいでスピンについての正しい理解が阻害されているような気もする.
また記事の途中にも説明するがスピンは粒子がもつ一つの自由度と捉えるべきだと考えている.}.
そのため電子は1つの小さな磁石のように振る舞うが,この磁気モーメント$\bvec{\mu}_s$は
ボーア磁子$\mu_B(=\frac{e\hbar}{2m})$と電子のg因子($\backsimeq 2$)
\footnote{おおよそ2であるが,2からのずれは異常磁気モーメントと呼ばれ量子電磁力学で説明される.}
を用いて次のように表される.
\begin{align}
  \bvec{\mu}_s = -g\mu_B\bvec{s} = -\frac{e}{m} \bvec{s}
\label{ボーア}
\end{align}
原子核と電子が起こす電磁相互作用のエネルギーは$\bvec{B}_{\text{eff}}と\bvec{\mu}_s$が平行のとき最も安定で,
反平行のとき最も不安定になるので$-\bvec{\mu}_s\cdot\bvec{B}_{\text{eff}}$の形で表される.
\footnote{例えば後藤憲一「詳解 電磁気学演習」p.187に導出が載っている.}.
したがって電子と原子核間の電磁相互作用のエネルギーは次の形で書かれる.
\begin{align}
  \mathcal{H} _{so} =  \frac{\mu_0}{4\pi}\frac{Ze^2}{m^2}\frac{1}{r^3}\bvec{s}\cdot \bvec{l}
  \label{古典}
\end{align}
古典的に導かれたスピン軌道相互作用は相対論的量子力学で導かれるものより2倍大きい.
これは電子の加速度運動の取り扱いが完全ではないことに依る.

この記事では相対論的量子力学を用いてスピン軌道相互作用を導出することを目的とする.
この記事において間違いや疑問などがある場合は質問部屋等で指摘して頂けるとありがたい.


\section{電磁場中の荷電粒子のハミルトニアン}
量子力学では系のハミルトニアンを基に議論を進めていくが,相対論的量子力学でも同様である.
そのため,先に非相対論的な電磁場中の荷電粒子のハミルトニアンを求めておく.
(この節では\cite{I}を参考にした)

真空中の電磁場内で運動する質量$m$, 電荷$q$の粒子が速度$\bvec{v}$で運動する時,
次のようなローレンツ力が粒子に働く.
\begin{align}
  \bvec{F}=q\left[\bvec{E}+(\bvec{v}×\bvec{B})\right]
\end{align}
$\bvec{E}と\bvec{B}$の代わりにベクトルポテンシャル$\bvec{A}$とスカラーポテンシャル$\phi$を用いると,
\begin{subequations}
  \begin{empheq}[left = {\empheqlbrace \,}, right = {}]{align}
  & \bvec{E} = \operatorname{grad} \phi -\frac{\partial \bvec{A}}{\partial t}\\
  & \bvec{B} = \operatorname{rot} \bvec{A}
  \end{empheq}
\end{subequations}
の関係から運動方程式は
\vspace{3pt}
\begin{subequations}
  \begin{empheq}[left = {}, right = { \empheqrbrace}]{align}
  & m\ddot{x}=-q\frac{\partial \phi}{\partial x}-q \frac{\partial A_x}{\partial t}
  +q\left[\dot{y}\left(\frac{\partial A_y}{\partial x}-\frac{\partial A_x}{\partial y}\right)
  -\dot{z}\left(\frac{\partial A_x}{\partial z}-\frac{\partial A_z}{\partial x}\right)\right]\\ 
  & m\ddot{y}=-q\frac{\partial \phi}{\partial y}-q \frac{\partial A_y}{\partial t}
  +q\left[\dot{z}\left(\frac{\partial A_z}{\partial y}-\frac{\partial A_y}{\partial z}\right)
  -\dot{x}\left(\frac{\partial A_y}{\partial x}-\frac{\partial A_x}{\partial z}\right)\right]\\
  & m\ddot{z}=-q\frac{\partial \phi}{\partial z}-q \frac{\partial A_z}{\partial t}
  +q\left[\dot{x}\left(\frac{\partial A_x}{\partial z}-\frac{\partial A_z}{\partial x}\right)
  -\dot{y}\left(\frac{\partial A_z}{\partial y}-\frac{\partial A_y}{\partial z}\right)\right]
  \end{empheq}
  \label{めんどい}
\end{subequations}
\vspace{3pt}
ここでは次で行う計算を分かりやすくするために成分ごとに書いた.

ハミルトニアン$\mathcal{H} $を求めるにはラグランジアン$\mathcal{L}$を求めて,
一般化運動量を定めなくてはならない.
ここでは天下りであるが先に$\mathcal{L}$を与えることにする.
\begin{align}
  \mathcal{L}&=\frac{m}{2}\dot{\bvec{r}}^2-q\phi+q\dot{\bvec{r}}\cdot\bvec{A}\\
  &=\frac{m}{2}(\dot{x}^2\dot+{y}^2+\dot{z}^2)-q\phi
  +q(\dot{x}A_x+\dot{y}A_y+\dot{z}A_z) \nonumber
\end{align}
このように$\mathcal{L}$をとるとEL方程式
\begin{align}
  \frac{d}{dt}\left(\frac{\partial \mathcal{L}}{\partial \dot{x}}\right)-\frac{\partial \mathcal{L}}{\partial x}=0, \cdots
\end{align}
が(\ref{めんどい})の各式と一致することが簡単な計算で分かる.
ただし微分の際に
\begin{align}
  \frac{d}{dt}A_x(x,y,z,t) = \frac{\partial A_x}{\partial t}+\dot{x}\frac{\partial A_x}{\partial x}
  +\dot{y}\frac{\partial A_x}{\partial y}+\dot{z}\frac{\partial A_x}{\partial z}
\end{align}
となることに注意しなければならない.

したがって一般化運動量$\bvec{p}$は
\begin{align}
  \bvec{p} =\frac{\partial \mathcal{L}}{\partial \dot{\bvec{r}}}= m\dot{\bvec{r}} +q \bvec{A}
\end{align}
となるので,ハミルトニアン$\mathcal{H}$は
\begin{align}
  \mathcal{H}&=\bvec{p}\cdot\dot{\bvec{r}}-\mathcal{L}\nonumber\\
&=\frac{m}{2}\dot{\bvec{r}}^2+q\phi\nonumber\\
&=\frac{1}{2m}(\bvec{p}-q\bvec{A})^2+q\phi
\end{align}となる.
すなわち電磁場中の荷電粒子のハミルトニアンは
\begin{align}
  \bvec{p} → \bvec{p}-q\bvec{A}, \mathcal{H} → \mathcal{H}-q\phi
\end{align}
とすればよいことがわかる.

\section{相対論的量子力学}
\subsection{クライン-ゴルドン方程式}
量子力学における方程式はハミルトニアン$\mathcal{H}$と運動量$\bvec{p}$を
\begin{align}
  \bvec{p} → -i\hbar\nabla, \mathcal{H} = E →  i\hbar \frac{\partial }{\partial t}
\label{量子化} 
\end{align}
と変換して波動関数$\psi(\bvec{r},t)$に作用させればよかったことを思い出そう
\footnote{そんなんでいいのか?と思ってるが上手くいってるからおk}.

例えばポテンシャル$V(\bvec{r})$に束縛された粒子におけるハミルトニアンは
\begin{align}
  \mathcal{H}= \frac{\bvec{p}^2}{2m} + V(\bvec{r})
\end{align}と書かれるが,この式に(\ref{量子化})の変換を施して波動関数に作用させると
\begin{align}
  i\hbar \frac{\partial \psi}{\partial t} = \left\{-\frac{\hbar^2}{2m}\Delta +V(\bvec{r})\right\}\psi
\end{align}
と時間に依らないシュレディンガー方程式が導出される.

さて,特殊相対論の帰結においてエネルギーは
\begin{align}
  E^2 = c^2\bvec{p}^2 + m^2c^4
  \label{E}
\end{align}
と表される.
この式に先ほどの量子化の手続きを行うとどうなるだろうか?
実際に行ってみると
\begin{align}
  -\hbar^2 \frac{\partial^2 }{\partial t^2}\psi = (-c^2\hbar^2\nabla^2+m^2c^4) \psi
\end{align}
ダランベルシアン$\Box = \Delta -\frac{1}{c^2}\frac{\partial^2 }{\partial t^2}$を用いて変形すると
\begin{align}
  \left(\Box - \frac{m^2c^2}{\hbar^2}\right) \psi = 0
  \label{クライン}
\end{align}
この方程式(\ref{クライン})を\textbf{クライン-ゴルドン方程式}
\footnote{実は先にシュレディンガーも同じ式を導いていたが後に説明するように不都合なことが起きて諦めてしまっている.}
と呼ぶ.
この式はローレンツ不変な形になっている.
しかしこの方程式はある欠点を抱えている.
シュレディンガー方程式の波動関数$\psi$は確率解釈が可能であったが,
クライン-ゴルドン方程式ではそれができないのである.

確率解釈とは$|\psi|^2$を粒子の存在確率だと思うことであるが,
そのためには規格化条件
\begin{align}
  \int |\psi|^2 d\bvec{r} = 1
\end{align}
が時間に依らずに成り立たなくではならないので
\begin{align}
  \frac{d}{dt} \int |\psi|^2 d\bvec{r} = 
  \int \left(\frac{\partial \psi^*}{\partial t}\psi + \psi^*\frac{\partial \psi}{\partial t}\right)d\bvec{r} = 0
\end{align}
が成り立たなくてはならない.しかしクライン-ゴルドン方程式は時間$t$について2階の微分方程式なので
$\psi,\frac{\partial \psi}{\partial t}$は自由に取ることが出来るが,これでは確率の保存則が成り立たなくなってしまう.
他にも理論から導かれる値が実験値と合わないなどクライン-ゴルドン方程式は1粒子の波動関数と考えるのは無理がある
\footnote{しかし場の量子論においては問題が解消され意味のある重要な式となる}.


\subsection{ディラック方程式}
\subsubsection{ディラック行列}
クライン-ゴルドン方程式が確率解釈出来ないことは方程式が
時間の2階微分を持っていることに由来した.
それでは特殊相対論のエネルギーの式(\ref{E})で
\begin{align}
  \frac{E}{c} = \pm \sqrt{\bvec{p}^2 + m^2c^2}
  \label{E'}
\end{align}
を考えればいいような気がしてくる.
しかしこのまま量子化の規則を用いると右辺の根号の部分はどう扱えばよいのであろうか?
考えられるものとしてはテイラー展開を行う方法であるが,方程式が無限回の微分を含む形になってしまい不都合である.
そのため右辺を次のように1次式に等しいと置いてみることにする.
\begin{align}
  \sqrt{\bvec{p}^2 + m^2c^2} = \alpha_1 p_1 + \alpha_2 p_2 + \alpha_3 p_3 +  \beta mc
  \label{展開}
\end{align}
ここでは$x,y,zの代わりに添え字1,2,3$を用いている.
実はこれが上手く行く方法である.
以下では正のエネルギー解を元に話を進めていく.
負のエネルギー解を簡単に捨てていいのかと思うが,
これについては正負どちらの解を選んでも良いことがのちに分かる.

(\ref{展開})の両辺を2乗して係数比較をすると$\alpha_1,\alpha_2,\alpha_3,\beta$は次式を満たさなくてはならないことが分かる.
\begin{subequations}
  \begin{empheq}[left = {}, right = {}]{align}
  &\alpha_1^2 = \alpha_2^2 = \alpha_3^2 = \beta ^2 = 1 \label{単位}\\
  &\alpha_i\alpha_k + \alpha_k\alpha_i = 0  (i,k=1,2,3   i\neq k) \label{反交換1}\\
  &\alpha_i\beta + \beta\alpha_i=0  (i=1,2,3) \label{反交換2}
  \end{empheq}
  \label{条件}
\end{subequations}

この関係式はスカラーでは満たすことが出来ない.
したがって$N×N$行列だと仮定してみよう.

反交換関係(\ref{反交換1})満たすのでその行列式の値の間には
\begin{align}
  \text{det}\ \alpha_i \cdot \text{det}\ \alpha_k = (-1)^N \text{det}\ \alpha_k \cdot \text{det}\ \alpha_i
\end{align}
の関係が成り立つ.\ $\beta$についても同様なので$N$は偶数であることがわかる.
それでは最も小さい$N=2$について考えてみよう.

2×2行列で(\ref{条件})を満たすような行列はまずパウリ行列が思いつくと思う.
パウリ行列とは以下の行列である.
\begin{equation}
  \sigma_1 = \left(
    \begin{array}{cc}
      0 & 1 \\ 
      1 & 0
    \end{array}
  \right), 
  \sigma_2 = \left(
    \begin{array}{cc}
      0 & -i \\ 
      i & 0
    \end{array}
  \right), 
  \sigma_3 = \left(
    \begin{array}{cc}
      1 & 0 \\ 
      0 & -1
    \end{array}
  \right)
\end{equation}
パウリ行列は以下の関係を満たす.
\begin{subequations}
  \begin{empheq}[left = {}, right = {}]{align}
  &(\sigma_j)^2 = \sigma_j^†\sigma_j= I  (Iは単位行列 \ j=1,2,3)\\
  &\sigma_j\sigma_k = -\sigma_k\sigma_j = i\varepsilon_{jkl}\sigma_l (j,k,l = 1,2,3)\label{パウリ2}
  \end{empheq}
  \label{パウリ}
\end{subequations}
(\ref{パウリ2})式より
\begin{align}
  \sigma_j\sigma_k + \sigma_k\sigma_j = 0
\end{align}
が成り立つので$\alpha_j = \sigma_j$として
(\ref{条件})が成り立つ$\beta$を見つければよい.
ここで$a_0,a_1,a_2,a_3$を任意の複素数として
\begin{align}
  \beta = a_0I + a_1\sigma_1 + a_2\sigma_2 + a_3\sigma_3
\end{align}
とすれば任意の2×2複素行列を表現することができる($I$は単位行列).
これが(\ref{反交換2})式を満たすとすると
\begin{align}
  a_0\sigma_j + a_jI = 0 
\end{align}
が$j=1,2,3$の全てで成り立つため,
$a_0=a_j=0$でないといけない.
しかしこのとき$\beta=0 $なので
(\ref{単位})式の$\beta^2 = 1$が成り立たなくなってしまう.そのためこのときは解がないことが分かる.

いまパウリ行列を例に出したが2×2行列のときは(\ref{条件})を満たす解が存在しないことが分かっている.

そのため少なくとも(\ref{条件})を満たすのは$N=4$以上かつ偶数である$N$次正方行列である.
次は$N=4$の4次正方行列で考えるが,このときは(\ref{条件})を満たす行列が存在することがすでに分かっている.

例えば

\begin{equation}
  \begin{aligned}
    &\alpha_1 = \left(
      \begin{array}{cc}
        0 & \sigma_1 \\
        -\sigma_1 & 0 
      \end{array}
      \right)
    = \left(
      \begin{array}{cccc}
        0 & 0 & 0 & 1 \\
        0 & 0 & 1 & 0 \\
        0 &-1 & 0 & 0 \\
      -1 & 0 & 0 & 0
      \end{array}
      \right), 
    \alpha_2 = \left(
      \begin{array}{cc}
        0 & \sigma_2 \\
        -\sigma_2 & 0 
      \end{array}
      \right)
    = \left(
      \begin{array}{cccc}
        0 & 0 & 0 & -i\\
        0 & 0 & i & 0 \\
        0 & i & 0 & 0 \\
       -i & 0 & 0 & 0
      \end{array}
      \right)\\
    &\alpha_3 = \left(
        \begin{array}{cc}
          0 & \sigma_3 \\
          -\sigma_3 & 0 
        \end{array}
        \right)
      = \left(
        \begin{array}{cccc}
          0 & 0 & 1 & 0 \\
          0 & 0 & 0 & -1\\
         -1 & 0 & 0 & 0 \\
          0 & 1 & 0 & 0
        \end{array}
        \right), 
      \beta = \left(
        \begin{array}{cc}
          I & 0 \\
          0 & -I 
        \end{array}
        \right)
      = \left(
        \begin{array}{cccc}
          1 & 0 & 0 & 0 \\
          0 & 1 & 0 & 0 \\
          0 & 0 &-1 & 0 \\
          0 & 0 & 0 &-1
        \end{array}
        \right)
  \end{aligned}
\label{ディラック表示}
\end{equation}

(\ref{条件})を満たす行列のことを\textbf{ディラック行列}
と呼び,(\ref{ディラック表示})を\textbf{ディラック表示}と呼ぶ.

そのため他にもディラック行列の表示は存在する.
例えば
\begin{equation}
  \begin{aligned}
    &\alpha_1 = \left(
      \begin{array}{cc}
        0 & \sigma_1 \\
        -\sigma_1 & 0 
      \end{array}
      \right)
    = \left(
      \begin{array}{cccc}
        0 & 0 & 0 & 1 \\
        0 & 0 & 1 & 0 \\
        0 &-1 & 0 & 0 \\
      -1 & 0 & 0 & 0
      \end{array}
      \right), 
    \alpha_2 = \left(
      \begin{array}{cc}
        0 & \sigma_2 \\
        -\sigma_2 & 0 
      \end{array}
      \right)
    = \left(
      \begin{array}{cccc}
        0 & 0 & 0 & -i\\
        0 & 0 & i & 0 \\
        0 & i & 0 & 0 \\
       -i & 0 & 0 & 0
      \end{array}
      \right)\\
    &\alpha_3 = \left(
        \begin{array}{cc}
          0 & \sigma_3 \\
          -\sigma_3 & 0 
        \end{array}
        \right)
      = \left(
        \begin{array}{cccc}
          0 & 0 & 1 & 0 \\
          0 & 0 & 0 & -1\\
         -1 & 0 & 0 & 0 \\
          0 & 1 & 0 & 0
        \end{array}
        \right), 
      \beta = \left(
        \begin{array}{cc}
          0 & I \\
          I & 0 
        \end{array}
        \right)
      = \left(
        \begin{array}{cccc}
          0 & 0 & 1 & 0 \\
          0 & 0 & 0 & 1 \\
          1 & 0 & 0 & 0 \\
          0 & 1 & 0 & 0 
        \end{array}
        \right)
  \end{aligned}
\label{ワイル表示}
\end{equation}
はワイル表示と呼ばれる.
他にもパウリ表示やマヨラナ表示などがある.
表示が異なったとしても物理的内容は全く変わらないので,
知りたい物理によって表示を使い分ける.

もちろん6次以上の正方行列を仮定しても\eqref{条件}を満たす行列は存在するが
ここでは立ち入らないことにする
\footnote{正直そこまで立ち入りたくないので,素粒子理論の人や数学者に
ゴリゴリ頑張って欲しい}.

\subsubsection{ディラック方程式}
さて,前節では(\ref{展開})式を正当化出来ることが分かった.
そのためこの式を用いて量子化を行うと
\begin{align}
  i\hbar \frac{\partial \psi}{\partial t} &= (\bvec{\alpha}\cdot\bvec{p} + \beta mc^2)\psi\nonumber\\
  &= \left[-ic\hbar
  \left(
    \alpha_1 \frac{\partial }{\partial x_1}+\alpha_2 \frac{\partial }{\partial x_2}+\alpha_3 \frac{\partial }{\partial x_3}
  \right)
  +\beta mc^2
  \right]\psi 
\label{ディラック}
\end{align}
この方程式を外場のない\textbf{ディラック方程式}と呼ぶ.
ディラック方程式はローレンツ共変で,確率解釈にも問題が起きない.
ここでディラック方程式が元にしているのは特殊相対論の式なので
ディラックの相対論的量子力学は正確には特殊相対論的量子力学である
ことに注意されたい.

ここで{(\ref{ディラック})}式の右辺を$\mathcal{H}$とおくと
\begin{align}
  \mathcal{H}\psi &= i\hbar \frac{\partial \psi}{\partial t}
\end{align}
と時間に依存するシュレディンガー方程式と似た形になるが,
ハミルトニアン$\mathcal{H}$の中身や波動関数$\psi$が4成分持つ点などが全く異なる.
以降このノートではディラック表示を用いる.


\subsubsection{保存量}
波動関数が4つの成分を持たなければいけないことは分かったがこの物理的意味はなんだろうか?
その意味を調べるために自由電子のディラック方程式の保存量を調べることにする.

ハイゼンベルクの運動方程式を思い出すと
\begin{align}
\frac{\partial A}{\partial t} = \frac{i}{\hbar}\left[\mathcal{H},A\right]  
\end{align}
であったので,物理量$A$が$\mathcal{H}$と可換
\begin{align}
  \left[\mathcal{H},A\right]  = 0
\end{align}
のとき$A$は保存量となる.

それでは,まず運動量$\bvec{p}$について考えると明らかに
\begin{align}
  \left[\mathcal{H},\bvec{p}\right] =0 
\end{align}
が成り立つ.
したがって運動量$\bvec{p}$
外場がないので当然である
\footnote{じつは位置$\bvec{r}$は保存量にはならない.
時間発展を考えると位置は光速で揺らいでいることが分かるので考えてみると面白いと思う.}
.

次は軌道角運動量$\bvec{L}_j =   \varepsilon_{ijk}x_kp_l\ (j,k,l=1,2,3)$
について考えると
\begin{align}
  \left[\mathcal{H},L_3\right] &= \left[\sum_j \alpha_jp_j+\beta m,\ x_1p_2-x_2p_1\right] \nonumber\\
  &=\sum_j \alpha_j\left([p_j,x_1]p_2-[p_jx_2]p_1\right) \nonumber\\
  &=i\hbar (-\alpha_1p_2+\alpha_2p_1)\ (\neq 0)\label{軌道}
\end{align}
ここで$\left(\because \left[x_i,p_l\right]=i\hbar\delta_{i,j}\right)$を用いた.
したがって軌道角運動量$\bvec{L}$は保存量ではない.

ここで新しく
\begin{equation}
\sigma'_j := \left(
\begin{array}{cc}
  \sigma_j & 0 \\
  0 & \sigma_j 
\end{array}
\right)
\end{equation}
という演算子を定義すると,この演算子の$\mathcal{H}$との交換関係は

\begin{equation}
  \begin{aligned}
    \left[\mathcal{H},\sigma'_k\right]&= \sum_j[\alpha_j,\sigma'_k]p_j + [\beta,\sigma'_k]m\\
  &= 2i \sum_{j,l} \varepsilon_{jkl} \left(
    \begin{array}{cc}
      0 & \sigma_l \\
      -\sigma_l & 0 \\
    \end{array}
    \right)p_j 
  \end{aligned}
\end{equation}
ここではパウリ行列の交換関係$[\sigma_j,\sigma_k]=2i\sum_{l}\varepsilon_{jkl}\sigma_l$を用いた.

したがって特に$\sigma'_3$との交換関係は
\begin{align}
  \left[\mathcal{H},\sigma'_3\right]=2i(\alpha_1p_2-\alpha_2p_1)
\end{align}
ここで$\bvec{s}:=\frac{\hbar}{2}\bvec{\sigma}$と定義すると
\begin{align}
  \left[\mathcal{H},L_3+s_3\right] =0
\end{align}
他の成分についても同様なので
\begin{align}
\left[\mathcal{H},\bvec{L}+\bvec{s}\right] =0
\end{align}
$\bvec{s}$はのちにスピン演算子だとわかるので
全角運動量$\bvec{J}=\bvec{L}+\bvec{s}$が保存量となる.
つまり相対論的量子力学の立場だと全角運動量が良い量子数となるのである.

\subsubsection{静止しているときのディラック方程式の解}
静止しているときのディラック方程式は$\bvec{p}=0$なので
\begin{align}
  i \frac{\partial }{\partial t} \psi(t)= \beta m \psi(t)
\end{align}
成分ごとに表示すると
\begin{subequations}
  \begin{empheq}[left = {\empheqlbrace}, right = {}]{align}
  & i \frac{\partial \psi_j}{\partial t} = m \psi_j (j=1,2)\\
  & i \frac{\partial \psi_k}{\partial t} = -m \psi_k (k=3,4)
  \end{empheq}
  \label{自由電子}
\end{subequations}
これらの解は
\begin{equation}
\psi = \left(\psi_1,\ \psi_2,\ \psi_3,\ \psi_4\right)=
\left(
\begin{array}{cccc}
  e^{-im} &  &  &  \\
   & e^{-im} &  &  \\
   &  & e^{im} &  \\
   &  &  & e^{im} 
\end{array}
\right)
\end{equation}
ここでは4つの波動関数をまとめて書いている.

波動関数が求まったが4成分ある意味について考えよう.
まずエネルギーについては
\begin{equation}
  E\psi = i \frac{\partial }{\partial t} \psi = \left(
  \begin{array}{cccc}
    m &  &  &  \\
     & m &  &  \\
     &  & -m &  \\
     &  &  & -m 
  \end{array}
  \right)\psi
\end{equation}
したがって正のエネルギー状態と負のエネルギー状態が存在している.
粒子はより低いエネルギーをとるので正のエネルギーを持つ粒子も光を放出しながら
低いエネルギーへどんどん遷移していってしまう.
しかし正エネルギーの電子は安定して存在している.

このことについてディラックは負のエネルギー解はすでに粒子が埋め尽くされていて,
フェルミの排他律によって負のエネルギーへの遷移が禁止されて粒子が安定して存在していると理解した.
この解釈によると真空状態と考えている状態は既に負のエネルギーの粒子が埋め尽くされているので
,この真空状態を\textbf{ディラックの海}
\footnote{高校生の頃はなんかかっこいいとテンションが上がっていたが,今ならなるほどなぁという感じだ.
またこの話はフェルミ粒子であれば同じように適用出来るので空孔理論と一般化して呼ばれて,
物性の人であれば聞いたことがあるであろうフェルミの海も空孔理論を用いてホールを
陽子として捉えている.筆者は場の量子論を勉強していないので知らないが反粒子は
時間に逆行している粒子としても捉えることが出来るらしいので誰か教えて欲しい}
と呼ぶ.
このディラックの海から粒子が励起した場合,空孔が出来ることになる.
この空孔のことを\textbf{反粒子}として捉える.
例えば電子の反粒子は陽電子である.
ディラックの海は不確定性原理からエネルギーが$2m$を超えることで
,粒子と反粒子が対生成と対消滅を繰り返して絶えず揺らいでいる(\textbf{真空編極}).

次に$s_3$を作用させてみると
\begin{equation}
 s_3\psi= \frac{\hbar}{2}\sigma'_3 \psi=\frac{\hbar}{2}\left(
\begin{array}{cccc}
  1 &  &  &  \\
  & -1 &  &  \\
  &  & 1 &  \\
  &  &  & -1  
\end{array}
\right)\psi
\end{equation}
したがって固有値は$+\frac{\hbar}{2}$と$-\frac{\hbar}{2}$となる.
これはまさにスピンのアップとダウンであるので,$s_3$は
実はスピン演算子であった.

流れを整理するとディラック方程式を導く際に波動関数は4成分持つ必要があるとわかったが,
その成分の一つの自由度はエネルギーの正負であった.
またもう一つの自由度はスピンのアップとダウンであることが今の議論でわかったことになる.
量子力学ではスピンを新しく入れてやる必要があったが,
ディラックの相対論的量子力学からは必然的に導かれるのである.

ここでディラックの海の話をするときに突然フェルミ粒子だと仮定していた.
これは場の量子化をするときにわかるが,ディラック方程式が従う粒子は
フェルミ粒子でなければいけないことがわかる.
そのためディラック方程式はフェルミ粒子が従う特殊相対論的量子力学の
方程式である.ボーズ粒子が従う相対論的な方程式は
マクスウェル方程式やプロカ方程式がある.


\section{電磁場中のディラック電子}
2章において,荷電粒子のハミルトニアンは
\begin{gather}
  \bvec{p} → \bvec{p}-q\bvec{A}, \mathcal{H} → \mathcal{H}-q\phi
  \label{変換}
\end{gather}
と変換すればよいことをみた.

そのため電磁場中の電子に対するディラック方程式は$q=-e$として(\ref{ディラック})式に(\ref{変換})式の変換を行うと
\begin{align}
  \left(i\hbar \frac{\partial }{\partial t}+e\phi \right)\psi 
  =\left[c\bvec{\alpha}\cdot\left(\bvec{p}+e\bvec{A}\right)+\beta mc^2\right]\psi
\label{電磁場}  
\end{align}
となる.

\subsection{大きい解,小さい解}
ここで定常解として
\begin{align}
  \psi(\bvec{r},t) = \exp\left(-i\frac{\varepsilon}{\hbar}t\right)\psi(\bvec{r})
\end{align}
を仮定すると,電磁場中の電子のディラック方程式は
\begin{align}
  \left(\varepsilon+e\phi -\beta mc^2\right)\psi(\bvec{r}) 
  =c\bvec{\alpha}\cdot\left(\bvec{p}+e\bvec{A}\right)\psi(\bvec{r})
  \label{定常解}
\end{align}
となる.
具体的に$\alpha と\beta$のディラック表示を用いて成分ごとに書くと
\begin{subequations}
  \begin{equation}
    \begin{aligned}
      \left(\varepsilon + e\phi -mc^2\right)
      \left(
        \begin{array}{c}
        \psi_1 \\
        \psi_2
      \end{array}
      \right)
      =c\bvec{\sigma}\cdot\left(\bvec{p}+e\bvec{A}\right)
      \left(
        \begin{array}{c}
        \psi_3 \\
        \psi_4
      \end{array}
      \right)
    \end{aligned}
    \label{1}
  \end{equation}
  
  \begin{equation}
    \begin{aligned}
      \left(\varepsilon + e\phi +mc^2\right)
      \left(
        \begin{array}{c}
        \psi_3 \\
        \psi_4
      \end{array}
      \right)
      =c\bvec{\sigma}\cdot\left(\bvec{p}+e\bvec{A}\right)
      \left(
        \begin{array}{c}
        \psi_1 \\
        \psi_2
      \end{array}
      \right)
    \end{aligned}
    \label{2}
  \end{equation}  
  \label{大小}
\end{subequations}  
ここで$\varepsilon = mc^2 + \varepsilon'$
とすると$\varepsilon'$は電子のエネルギー$\varepsilon$から静止エネルギー$mc^2$を引いた値なので$e\phi\ll mc^2$のとき,
非相対論的極限において
\begin{subequations}
  \begin{equation}
    \varepsilon + e\phi -mc^2 = \varepsilon' +e\phi \ll mc^2
  \end{equation}

  \begin{equation}
    \varepsilon + e\phi +mc^2 \backsimeq 2mc^2
  \end{equation}
\end{subequations}

また$c\bvec{\sigma}\cdot\left(\bvec{p}+e\bvec{A}\right)は 
mcv$くらいの大きさを与えるので,
\eqref{大小}式で($\psi_3,\psi_4$)成分は($\psi_1,\psi_2$)成分と
比べて$c$倍くらい小さいことが分かる.

\subsection{スピンの磁気モーメント}
電磁場中の電子のディラック方程式\eqref{電磁場}式は
\begin{align}
  \left[\left(i\hbar \frac{\partial }{\partial t}+e\phi \right)
  -c\bvec{\alpha}\cdot\left(\bvec{p}+e\bvec{A}\right)-\beta mc^2\right]\psi=0
\end{align}
とも書けるが,この式に左から$ \left[\left(i\hbar \frac{\partial }{\partial t}+e\phi \right)
+c\bvec{\alpha}\cdot\left(\bvec{p}+e\bvec{A}\right)+\beta mc^2\right]$をかけると
\begin{align}
  \left[\left(i\hbar \frac{\partial }{\partial t}+e\phi \right)^2
  -\left(c\bvec{\alpha}\cdot\left(\bvec{p}+e\bvec{A}\right)-\beta mc^2\right)^2+i\hbar ce\bvec{\alpha}\cdot\bvec{E}\right]\psi=0
\label{がんばる}
\end{align}となる.
ここで第3項の導出は
\begin{align}
  &\left(i\hbar \frac{\partial }{\partial t}+e\phi \right)\left(c\bvec{\alpha}\cdot\left(\bvec{p}+e\bvec{A}\right)-\beta mc^2\right)-
  \left(c\bvec{\alpha}\cdot\left(\bvec{p}+e\bvec{A}\right)-\beta mc^2\right)\left(i\hbar \frac{\partial }{\partial t}+e\phi \right) \nonumber\\
  =&\ i\hbar ce\left(\sum_j\alpha_j \frac{\partial A_j}{\partial t}\right) 
  +i\hbar ce \left(\sum_j\alpha_j \frac{\partial \phi}{\partial x_j}\right) \nonumber\\
  =&\ i\hbar ce\sum_j\alpha_jE_j
  =\ i\hbar ce\bvec{\alpha}\cdot\bvec{E}
\end{align}

また\eqref{がんばる}式第2項は
\begin{align}
  \left(c\bvec{\alpha}\cdot\left(\bvec{p}+e\bvec{A}\right)-\beta mc^2\right)^2 
  &=\left[c\sum_j\left(-i\hbar \frac{\partial }{\partial x_j}+eA_j\right)+\beta mc^2\right]^2 \nonumber\\
  &=c^2\sum_{j,k}\alpha_j\alpha_k\left(-i\hbar \frac{\partial }{\partial x_j}+eA_j\right)\left(-i\hbar \frac{\partial }{\partial x_k}+eA_k\right) \nonumber\\
   &+2mc^3\left[\sum_j\left(\alpha_j\beta+\beta\alpha_j\right)\left(-i\hbar \frac{\partial }{\partial x_j}+eA_j\right)\right] + \beta^2m^2c^4
\end{align}
ここで$\alpha_1,\alpha_2,\alpha_3,\beta$には\eqref{条件}の関係があるので
\begin{align}
  (\eqref{電磁場}第2項)&=c^2\left(\bvec{p}+e\bvec{A}\right)^2+m^2c^4 + W\\
  W &= c^2\sum_{j\neq k}\alpha_j\alpha_k\left(-i\hbar e \frac{\partial A_k}{\partial x_j}\right) \nonumber\\
  &=-i\hbar c^2e\left[\left(\alpha_1\alpha_2 \frac{\partial A_2}{\partial x_1}+
  \alpha_2\alpha_1 \frac{\partial A_1}{\partial x_2}\right) + \cdots \right] \nonumber\\
  &=-i\hbar c^2e\left[\alpha_1\alpha_2\left(\frac{\partial A_2}{\partial x_1}-\frac{\partial A_1}{\partial x_2}\right) + \cdots\right]\nonumber \\
  &=-i\hbar c^2e\left(\alpha_1\alpha_2B_3+\alpha_2\alpha_3B_1+\alpha_3\alpha_1B_2\right)
\end{align}
ここで$\alpha_1\alpha_2=i\sigma'_3,\ \alpha_2\alpha_3=i\sigma'_1,\ \alpha_3\alpha_1=i\sigma'_2$となることを
用いると
\begin{align}
  W = \hbar c^2e\bvec{\sigma'}\cdot\bvec{B}
\end{align}

結局\eqref{がんばる}式は次のように書かれる.
\begin{align}
  \left[\left(i\hbar \frac{\partial }{\partial t}+e\phi \right)^2-c^2\left(\bvec{p}+e\bvec{A}\right)^2-m^2c^4
  +i\hbar ce\bvec{\alpha}\cdot\bvec{E}-\hbar c^2e\bvec{\sigma'}\cdot\bvec{B}\right]\psi(\bvec{r},t)=0
\end{align}
この式においても\eqref{定常解}の定常解を仮定すると
\begin{align}
  \left[\left(\varepsilon+e\phi \right)^2-c^2\left(\bvec{p}+e\bvec{A}\right)^2-m^2c^4
  +i\hbar ce\bvec{\alpha}\cdot\bvec{E}-\hbar c^2e\bvec{\sigma'}\cdot\bvec{B}\right]\psi(\bvec{r})=0
\label{がんばった}
\end{align}
ここで非相対論的極限を考えると
\begin{align}
  \left(\varepsilon+e\phi\right)^2-m^2c^4&=\left(\varepsilon'+e\phi+mc^2\right)^2-m^2c^4\nonumber\\
&=2mc^2\left(\varepsilon'+e\phi\right) + \left(\varepsilon'+e\phi\right)^2 \nonumber\\
&\simeq 2mc^2\left(\varepsilon'+e\phi\right)
\end{align}
となる.したがって\eqref{がんばった}式は成分ごとに表記すると
\begin{subequations}
  \begin{equation}
    \begin{aligned}
      \left[2mc^2\left(\varepsilon'+e\phi\right)-c^2\left(\bvec{p}+e\bvec{A}\right)^2-c^2e\hbar\bvec{\sigma}\cdot\bvec{B}\right]
      \left(
        \begin{array}{c}
        \psi_1 \\
        \psi_2
      \end{array}
      \right)
      +ic\hbar e\bvec{\sigma}\cdot\bvec{E}
      \left(
        \begin{array}{c}
        \psi_3 \\
        \psi_4
      \end{array}
      \right)=0
    \end{aligned}
    \label{強い}
  \end{equation}
  
  \begin{equation}
    \begin{aligned}
      \left[2mc^2\left(\varepsilon'+e\phi\right)-c^2\left(\bvec{p}+e\bvec{A}\right)^2-c^2e\hbar\bvec{\sigma}\cdot\bvec{B}\right]
      \left(
        \begin{array}{c}
        \psi_3 \\
        \psi_4
      \end{array}
      \right)
      +ic\hbar e\bvec{\sigma}\cdot\bvec{E}
      \left(
        \begin{array}{c}
        \psi_1 \\
        \psi_2
      \end{array}
      \right)=0
    \end{aligned}
  \end{equation}
\end{subequations} 
ここで全体を$-2mc^2$で割って,小さい成分の項($\psi_3,\psi_4$)をムシすると\eqref{強い}は
\begin{equation}
  \left[\frac{1}{2m}\left(\bvec{p}+e\bvec{A}\right)^2+\frac{e\hbar}{2m}\bvec{\sigma}\cdot\bvec{B}-e\phi\right]
  \left(
    \begin{array}{c}
      \psi_1\\
      \psi_2
    \end{array}
  \right)=\varepsilon'
  \left(
    \begin{array}{c}
      \psi_1\\
      \psi_2
    \end{array}
  \right)
  \label{勝ち}
\end{equation}
となる.

ここで$  \left[\frac{1}{2m}\left(\bvec{p}+e\bvec{A}\right)^2-e\phi\right]$は
質量$m, 電荷-eの荷電粒子が電磁場\bvec{A}, \phi$で運動している場合の
シュレディンガー方程式のハミルトニアンである.
そのため相対論的量子力学に移ると
$\frac{e\hbar}{2m}\bvec{\sigma}\cdot\bvec{B}$の項が付け加わることになる.

これは$\mu_B=-\frac{e\hbar}{2m}$として
\begin{align}
  \frac{e\hbar}{2m}\bvec{\sigma}\cdot\bvec{B}
  =-\frac{2\mu_B}{\hbar} \bvec{s}\cdot \bvec{B}
\end{align}
と書き換えることが出来る.
これは\eqref{ボーア}式と一致している.
そのため電子の$g$因子は$\hbar=1$としたとき$g=2$となる.
以上のことは,電子はスピンをもつことにより
$-\frac{2\mu_B}{\hbar} $の磁気モーメントを持ち,
それと磁場が相互作用してゼーマンエネルギーが生じることを示している.

\subsection{スピン軌道相互作用}
さて,いよいよ本ノートの目標であるスピン軌道相互作用の項を導出しよう.
前節では$\psi_3,\psi_4$を無視していたが,次は無視しないで取り扱ってみる.
簡単のため磁場がない($\bvec{A}=0$)ときを考える.
このとき$\varepsilon+e\phi+mc^2\simeq 2mc^2$として得られる\eqref{2}の近似式は
\begin{equation}
    \begin{aligned}
      \left(
        \begin{array}{c}
        \psi_3 \\
        \psi_4
      \end{array}
      \right)
      =\frac{1}{2mc}\bvec{\sigma}\cdot\bvec{p}
      \left(
        \begin{array}{c}
        \psi_1 \\
        \psi_2
      \end{array}
      \right)
    \end{aligned}
  \end{equation}  
これを\eqref{強い}式に代入して$-2mc^2$で割ると
\begin{equation}
    \begin{aligned}
      -\frac{ie\hbar}{4m^2c^2}
      \left(\bvec{\sigma}\cdot\bvec{E}\right)
      \left(\bvec{\sigma}\cdot\bvec{p}\right)
      \left(
        \begin{array}{c}
        \psi_1 \\
        \psi_2
      \end{array}
      \right)
  \end{aligned}
  \label{卵}
\end{equation}
という項が\eqref{勝ち}式の左辺に追加される.

ここでパウリ行列の性質を用いると
\begin{align}
  \left(\bvec{\sigma}\cdot\bvec{E}\right)\left(\bvec{\sigma}\cdot\bvec{p}\right)
  &=\left(\sigma_1E_1+\sigma_2E_2+\sigma_3E_3\right)
  \left(\sigma_1p_1+\sigma_2p_2+\sigma_3p_3\right) \nonumber\\
  &=E_1p_1+E_2p_2+E_3p_3+\sigma_1\sigma_2\left(E_1p_2-E_2p_1\right)+
  \sigma_2\sigma_3\left(E_2p_3-E_3p_2\right)+\sigma_3\sigma_1\left(E_3p_1-E_1p_3\right) \nonumber\\
  &=\left(\bvec{E}\cdot\bvec{p}\right)+i\left(\bvec{\sigma}\cdot\left[\bvec{E}×\bvec{p}\right]\right)
\end{align}となるので
\eqref{卵}は
\begin{equation}
    \begin{aligned}
      \left[
        -\frac{ie\hbar}{4m^2c^2}\left(\bvec{E}\cdot\bvec{p}\right)
        +\frac{e\hbar}{4m^2c^2}\left(\bvec{\sigma}\cdot\left[\bvec{E}×\bvec{p}\right]\right)
      \right]
      \left(
        \begin{array}{c}
        \psi_1 \\
        \psi_2
      \end{array}
      \right)
  \end{aligned}
  \label{よくやった}
\end{equation}
ここで原子のs軌道などの中心力場では
\begin{align}
  \bvec{E} = -\ \text{grad}\ \phi(\bvec{r}) =-\bvec{r}\left(\frac{1}{r}\frac{d\phi}{dr}\right)
\end{align}
となるので\eqref{よくやった}第2項は
\begin{align}
  \mathcal{H}_{\text{so}} &= \frac{e\hbar}{4m^2c^2}\left(\bvec{\sigma}\cdot\left[\bvec{E}×\bvec{p}\right]\right) \nonumber\\
&=-\frac{e\hbar}{4m^2c^2}\left(\frac{1}{r}\frac{d\phi}{dr}\right)
\left(\bvec{\sigma}\cdot\left[\bvec{r}×\bvec{p}\right]\right) \nonumber\\
&=-\frac{e}{2m^2c^2}\left(\frac{1}{r}\frac{d\phi}{dr}\right)\left(\bvec{s}\cdot\bvec{l}\right)
\end{align}
ここで点電荷$Ze$がつくるスカラーポテンシャル$\phi$は
\begin{align}
  \phi = \frac{1}{4\pi \varepsilon_0}\frac{Ze}{r}
\end{align}
と表せるので
\begin{align}
  \mathcal{H}_{\text{so}} &= \frac{1}{4\pi\varepsilon_0}\frac{Ze^2}{2m^2c^2}\frac{1}{r^3}\left(\bvec{s}\cdot\bvec{l}\right) \nonumber\\
  &= \frac{\mu_0}{4\pi}\frac{Ze^2}{2m^2}\frac{1}{r^3}\left(\bvec{s}\cdot\bvec{l}\right)
\end{align}
これは第1節はじめにで導いた古典的なスピン軌道相互作用\eqref{古典}の$1/2$倍になっていて,
ディラック方程式を用いてスピン軌道相互作用を導出できたことになる
\footnote{触れなかったが\eqref{よくやった}第1項はダーウィン項と呼ばれ,
位置$\bvec{r}$の時間発展を考えたときにわかる運動によって引き起こされる電場の乱れである}.

\section{まとめ}
このノートではディラック方程式を導出して,
電磁場中の電子について考察することでスピン軌道相互作用を導出した.
その際,古典的に考えたスピン軌道相互作用とズレについても修正された.
またディラック方程式を考えることでスピンの概念が必然的に導入され,
そのスピンが磁気モーメントを持つことを確認した.
ここで注意しなければならないのはディラック方程式が従うのはフェルミ粒子であって,
ボーズ粒子には言及していない.
そのためボゾンについて知りたければもう少し場の量子論に踏み込んで話をしなければならない.


\section{さいごに}
今回は指定枠で記事を書かせて頂きました.
物性の話の中でスピン軌道相互作用は相対論的量子力学を用いて導出される,
としか記述されないのでいつか導出してやろうと考えていたのですが,今回
丁度いい機会だったのでまとめさせてもらいました.
ほんとは4元ベクトルを用いて記述した方がすっきりと書けることや,
ヘリシティやゲージ原理などもっと詰めたい部分もあったが,今回は載せられませんでした.
まだ勉強不足な所もあり,初めにも書きましたがノートに間違いなどが含まれていると思います.
改訂版を物理の会のサイトにアップロードしたいのでなにかあった場合は連絡を下さい.
さて,相対論的量子力学に興味がある人は
\cite{神}は初学者にむけてとても分かりやすく書いてるのでお勧めしたい.
また\cite{坂本}は相対論的量子力学に留まらずに対称性などを意識しながら
場の量子論へと発展させていくとても分かりやすい本です.元気がある素粒子理論志望の人が
学部3,4年くらいのゼミので使う本としては最適だと思いました.

今後は個人的に興味のある超伝導や磁性などの話題について
まとめたノートを書きたいと思っています.

思ったより長くなってしまいましたが,最後まで読んで頂いてありがとうございます.




%参考文献リスト
\begin{thebibliography}{99}
  \bibitem{II}小出昭一郎『量子力学(II)』第39版(裳華房,2015)
  \bibitem{soi}江藤幹雄,固体物理43『半導体中のスピン軌道相互作用入門(その1)』(2008)
  \bibitem{I}小出昭一郎『量子力学(I)』第50版(裳華房,2020)
  \bibitem{詳解}後藤憲一『詳解量子力学演習』初版第26刷(共立出版,2015)
  \bibitem{神}川村嘉春『相対論的量子力学』初版(裳華房,2012)
  \bibitem{坂本}坂本眞人『場の量子論 不変性と自由場を中心にして』初版(裳華房,2014)
  \bibitem{授業}山田將樹 相対論的量子力学の授業ノート(2023年度7セメスター)
\end{thebibliography}




\end{document}
