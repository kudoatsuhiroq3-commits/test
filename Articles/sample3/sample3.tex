


\section{サンプル}
サンプルサンプルサンプルサンプルサンプルサンプルサンプルサンプル
サンプルサンプルサンプルサンプルサンプルサンプルサンプルサンプル
サンプルサンプルサンプルサンプルサンプルサンプルサンプルサンプル
サンプルサンプルサンプルサンプルサンプルサンプルサンプルサンプル
サンプルサンプルサンプルサンプルサンプルサンプルサンプルサンプル
\subsection{サンプル!}
サンプルサンプルサンプルサンプルサンプルサンプルサンプルサンプル
サンプルサンプルサンプルサンプルサンプルサンプルサンプルサンプル
\subsubsection{サンプル...!}
サンプルサンプルサンプルサンプルサンプルサンプルサンプルサンプル


サンプルサンプルサンプルサンプルサンプルサンプルサンプルサンプル
サンプルサンプルサンプルサンプルサンプルサンプルサンプルサンプル
サンプルサンプルサンプルサンプルサンプルサンプルサンプルサンプル
サンプルサンプルサンプルサンプルサンプルサンプルサンプルサンプル
サンプルサンプルサンプルサンプルサンプルサンプルサンプルサンプル
\subsection{サンプル!!!}
サンプルサンプルサンプルサンプルサンプルサンプルサンプルサンプル
サンプルサンプルサンプルサンプルサンプルサンプルサンプルサンプル
サンプルサンプルサンプルサンプルサンプルサンプルサンプルサンプル


サンプルサンプルサンプルサンプルサンプルサンプルサンプルサンプル
サンプルサンプルサンプルサンプルサンプルサンプルサンプルサンプル
サンプルサンプルサンプルサンプルサンプルサンプルサンプルサンプル
サンプルサンプルサンプルサンプルサンプルサンプルサンプルサンプル
サンプルサンプルサンプルサンプルサンプルサンプルサンプルサンプル
サンプルサンプルサンプルサンプルサンプルサンプルサンプルサンプル
サンプルサンプルサンプルサンプルサンプルサンプルサンプルサンプル
サンプルサンプルサンプルサンプルサンプルサンプルサンプルサンプル


サンプルサンプルサンプルサンプルサンプルサンプルサンプルサンプル
サンプルサンプルサンプルサンプルサンプルサンプルサンプルサンプル
サンプルサンプルサンプルサンプルサンプルサンプルサンプルサンプル
サンプルサンプルサンプルサンプルサンプルサンプルサンプルサンプル
サンプルサンプルサンプルサンプルサンプルサンプルサンプルサンプル
サンプルサンプルサンプルサンプルサンプルサンプルサンプルサンプル
サンプルサンプルサンプルサンプルサンプルサンプルサンプルサンプル
サンプルサンプルサンプルサンプルサンプルサンプルサンプルサンプル

\section{サンプル?}
サンプルサンプルサンプルサンプルサンプルサンプルサンプルサンプル
サンプルサンプルサンプルサンプルサンプルサンプルサンプルサンプル
サンプルサンプルサンプルサンプルサンプルサンプルサンプルサンプル
サンプルサンプルサンプルサンプルサンプルサンプルサンプルサンプル
サンプルサンプルサンプルサンプルサンプルサンプルサンプルサンプル
\subsection{サンプル...}
サンプルサンプルサンプルサンプルサンプルサンプルサンプルサンプル
サンプルサンプルサンプルサンプルサンプルサンプルサンプルサンプル
サンプルサンプルサンプルサンプルサンプルサンプルサンプルサンプル


サンプルサンプルサンプルサンプルサンプルサンプルサンプルサンプル
サンプルサンプルサンプルサンプルサンプルサンプルサンプルサンプル
サンプルサンプルサンプルサンプルサンプルサンプルサンプルサンプル
サンプルサンプルサンプルサンプルサンプルサンプルサンプルサンプル
サンプルサンプルサンプルサンプルサンプルサンプルサンプルサンプル
サンプルサンプルサンプルサンプルサンプルサンプルサンプルサンプル
サンプルサンプルサンプルサンプルサンプルサンプルサンプルサンプル
サンプルサンプルサンプルサンプルサンプルサンプルサンプルサンプル


サンプルサンプルサンプルサンプルサンプルサンプルサンプルサンプル
サンプルサンプルサンプルサンプルサンプルサンプルサンプルサンプル
サンプルサンプルサンプルサンプルサンプルサンプルサンプルサンプル

サンプルサンプルサンプルサンプルサンプルサンプルサンプルサンプル
サンプルサンプルサンプルサンプルサンプルサンプルサンプルサンプル
サンプルサンプルサンプルサンプルサンプルサンプルサンプルサンプル
サンプルサンプルサンプルサンプルサンプルサンプルサンプルサンプル

\let\oldaddcontentsline\addcontentsline %消さないで!!
\renewcommand{\addcontentsline}[3]{} %消さないで!!
\begin{thebibliography}{99}
\item[]本稿では主に\cite{bibunsho}の内容を参考にした. 
どの章も\LaTeX を使った文書作成をする上で知っていて損はないが,特に第17章はこのpdfのようなものを作る上で非常に参考になる. 
\bibitem{bibunsho}奥村晴彦,黒木裕介『\LaTeX 美文書作成入門 第9版』技術評論社, 2023.
\item[] こんな感じで参考文献内に文章を入れられるらしい。すごいですね。hrefを使えば以下のようにもできるらしい。
\bibitem{M.Ozawa}\href{https://journals.aps.org/pra/abstract/10.1103/PhysRevA.67.042105}{M. Ozawa, Universally valid reformulation of the Heisenberg uncertainty principle on noise and  disturbance in measurement, Phys. Rev A \textbf{67}, 042105 (2003).}
\end{thebibliography}
\let\addcontentsline\oldaddcontentsline %消さないで!!