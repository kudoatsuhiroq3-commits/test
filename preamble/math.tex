

% 数式, 米国数学会が開発したのがamsmath, フォントを使うためのパッケージがamssymbs
\usepackage{amsmath,amssymb}
%定理環境、コピペ
\usepackage{amsthm}
\theoremstyle{definition}
\newtheorem{dfn}{定義}[section]
\newtheorem{prop}{命題}[section]
\newtheorem{lem}{補題}[section]
\newtheorem{thm}{定理}[section]
\newtheorem*{thm*}{定理}
\newtheorem{cor}{系}[section]
\newtheorem{rem}{注意}[section]
\newtheorem*{rem*}{注意}
\newtheorem{fact}{事実}[section]
\newtheorem{e.g.}{例}[section]

%証明終わりの四角
\renewcommand{\qedsymbol}{$\square$}

%花文字, RSFS, Ralph Smith's Formal Script
\usepackage{mathrsfs}

% 数式太文字
\usepackage{bm}

%proofカスタム, 適宜変更
\renewcommand{\proofname}{\textbf{証明}}

%\label{}して\ref, \eqref で参照した数式のみに数式番号が振られるようにしてる
%だからequationやalignを*なしで使っても参照しない限り式番号はつかない
\usepackage{mathtools}
\mathtoolsset{showonlyrefs=true}
%2024/10/3追加(工藤英哲さん用のパッケージ)
\usepackage{empheq}